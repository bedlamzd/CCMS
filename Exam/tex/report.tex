\documentclass[a4paper, 12pt]{article}

\usepackage{cmap}
\usepackage[T2A]{fontenc}
\usepackage[utf8]{inputenc}
\usepackage[english, russian]{babel}

% Для математики
\usepackage{amsmath}

% Для картинок
\usepackage{graphicx}
\usepackage{float}
\usepackage{wrapfig}
\usepackage[caption=false]{subfig}

% Оформление документа
\usepackage{indentfirst}
\usepackage{geometry}
\geometry{right=20mm}
\geometry{left=20mm}
\geometry{top=20mm}
\geometry{bottom=20mm}

% Пути до файлов
\graphicspath{{../src/img/}}

% Вставки кода
\usepackage{pythontex}
\pyc{data_file = '../src/data.txt'} % Путь до файла с дано
\begin{pycode}
with open(data_file, 'r') as data:
    K1, K2, T1 = data.readline().split()
print(f'$K_1={K1}$', f'$K_2={K2}$', f'$T_1={T1}$', sep=r'\\')
\end{pycode}


% Document specific settings
\graphicspath{{./img/}}


\begin{document}
    \begin{titlepage}
	\begin{center}
		\textit{МИНИСТЕРСТВО ОБРАЗОВАНИЯ И НАУКИ\\
			РОССИЙСКОЙ ФЕДЕРАЦИИ}
		\vspace{1ex}

		федеральное государственное бюджетное образовательное учреждение\\
		высшего профессионального образования
		\vspace{1ex}

		\textbf{САНКТ-ПЕТЕРБУРГСКИЙ НАЦИОНАЛЬНЫЙ ИССЛЕДОВАТЕЛЬСКИЙ УНИВЕРСИТЕТ ИТМО}
		\vspace{13ex}

		Лабораторная работа №4\\
		<<Анализ и моделирование систем с цифровым ПД-регулятором>>\\
		по дисциплине <<Компьютерное управление мехатронными системами>>\\
	\end{center}
	\vspace{15em}
	\begin{flushright}
		\noindent
		Выполнил:\\
		студент гр. R3425\\
		Борисов М. В.\\
		Преподаватель:\\
		Ловлин С. Ю.
	\end{flushright}
	\vfill
	\begin{center}
		\largeСанкт-Петербург\\
		2020 г.\\
	\end{center}
\end{titlepage}

    \setcounter{page}{2}
    
    \section{Задание}
        \begin{enumerate}
        \item Настроить цифровую систему управления на заданный оптимум.
        \item Промоделировать систему на единичный скачок задания
        \item Найти время переходного процесса и перерегулирование
        \end{enumerate}
        
    \section{Дано}
        \noindent Биномиальный оптимум
        \begin{flalign*}
        &W_\text{ОУ} = \dfrac{1}{s\left(T_1 s + 1\right)}&&\\
        &T_0 = 0.01&&\\
        &T_1 = 2T_0&&
        \end{flalign*}

    \section{Выполнение работы}
        \subsection{Вывод регулятора}
            \noindent$W_\text{ОУ} = \dfrac{1}{s\left(T_1 s + 1\right)}$ --- передаточная функция объекта управления\\
            $W_{\mbox{рс}}(s)=\dfrac{1}{3T_\mu s(T_\mu s+1)}$ --- передаточная функция разомкнутой системы.\\
            
            \[
                W_{\mbox{рег}}(s)=\dfrac{W_{\mbox{рс}}(s)}{W_{\mbox{оу}}(s)}=
                \dfrac{\cfrac{1}{3T_\mu s(T_\mu s+1)}}{\cfrac{1}{s(T_1s+1)}}=
                \dfrac{1}{3T_\mu}\dfrac{T_1 s+1}{T_\mu s +1}
            \]
            
            Видно, что при такой настройке в зависимости от выбора постоянной времени $ T_\mu $ возможно получить П- и ПД-регулятор. ПД-регулятор обладает лучшим быстродействием, чем П-регулятор, хотя производить расчёт таким регулятором вычислительно затратнее.
            
            Выберем ПД-регулятор и примем $ T_\mu = \dfrac{T_0}{2} $, таким образом получаем следующие коэффициенты:
            $$ K_{pa} = \dfrac{1}{3T_\mu} $$
            $$ K_{da} = T_1 $$
            $$ K_p = K_{pa} $$
            $$ K_d = \dfrac{1}{\exp\left(\dfrac{T_0}{K_{da}}\right) - 1} $$
        \subsubsection{Моделирование}
            Известно, что для биномиального оптимума время переходного процесса должно составлять примерно $ t_\text{пп} = 6.6T_\mu = 0.033 \text{с} $, а перерегулирование $ \Delta h = 0.43\% $. Значит в результате моделирования мы должны получить соответствующие значения для аналоговой системы.
            \begin{figure}[H]
                \centering\includegraphics[width=\linewidth, height=\textheight, keepaspectratio]{exam}
                \caption{Модель настраиваемой системы}
            \end{figure}
            \begin{figure}[H]
                \centering\includegraphics[scale=0.8]{graph}
                \caption{Результаты моделирования}
            \end{figure}
            На графике отмечены время завершения переходного процесса и перерегулирование, полученные в результате моделирования. Таким образом $ t_\text{пп} = 0.033,\,\Delta h = 0.4\% $, что соответствует теоретически ожидаемым значениям.
\end{document}