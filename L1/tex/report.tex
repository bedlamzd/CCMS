\documentclass[a4paper, 12pt]{article}

\usepackage{cmap}
\usepackage[T2A]{fontenc}
\usepackage[utf8]{inputenc}
\usepackage[english, russian]{babel}

% Для математики
\usepackage{amsmath}

% Для картинок
\usepackage{graphicx}
\usepackage{float}
\usepackage{wrapfig}
\usepackage[caption=false]{subfig}

% Оформление документа
\usepackage{indentfirst}
\usepackage{geometry}
\geometry{right=20mm}
\geometry{left=20mm}
\geometry{top=20mm}
\geometry{bottom=20mm}

% Пути до файлов
\graphicspath{{../src/img/}}

% Вставки кода
\usepackage[gobble=auto]{pythontex}
\pyc{data_file = '../src/data.txt'} % Путь до файла с дано
\begin{pycode}
with open(data_file, 'r') as data:
    K1, K2, T1 = data.readline().split()
\end{pycode}

\begin{document}
	\begin{titlepage}
	\begin{center}
		\textit{МИНИСТЕРСТВО ОБРАЗОВАНИЯ И НАУКИ\\
			РОССИЙСКОЙ ФЕДЕРАЦИИ}
		\vspace{1ex}
		
		федеральное государственное бюджетное образовательное учреждение\\
		высшего профессионального образования
		\vspace{1ex}
		
		\textbf{САНКТ-ПЕТЕРБУРГСКИЙ НАЦИОНАЛЬНЫЙ ИССЛЕДОВАТЕЛЬСКИЙ УНИВЕРСИТЕТ ИТМО}
		\vspace{13ex}
		
		Лабораторная работа №1\\
		<<Оптимумы линейных систем>>\\
		по дисциплине <<Компьютерное управление мехатронными системами>>\\
	\end{center}
	\vspace{15em}
	\begin{flushright}
		\noindent
		Выполнил:\\
		студент гр. R3425\\
		Борисов М. В.\\
		Преподаватель:\\
		Ловлин С. Ю.
	\end{flushright}
	\vfill
	\begin{center}
		\largeСанкт-Петербург\\
		2020 г.\\
	\end{center}
\end{titlepage}
	\section{Цель работы}
		\begin{enumerate}
			\item Исследовать характеристики системы настроенной на линейный, технический, симметричный, биномиальный оптимумы и астатизм третьего порядка.
			\item Исследовать реакцию системы на константное, линейное и квадратичное воздействие при разных оптимумах.
            \item Определить порядок астатизма системы настроенной на разные оптимумыю.
		\end{enumerate}
	\section{Дано}
		Для всех оптимумов заданы параметры $T_1=0.1, T_2=T_{\mu}=0.005, K_{\text{об}}=10$
		
		Входные воздействия:
		\begin{align*}
			  y &= A\\
			  y &= vt\\
			  y &= \frac{at^2}{2}\mbox{, где } A = 7, v=5, a=10
		\end{align*}
	\section{Выполнение работы}
		\begin{figure}[H]
			\centering
			\includegraphics[width=\textwidth]{sim_model}
			\caption{Модель simulink}
		\end{figure}
		\newpage
		\subsection{Линейный оптимум}
			\subsubsection{Переходная функция}
				$$W(s)=\frac{1}{Ts+1}$$
				\begin{figure}[H]
					\centering
					\includegraphics{linear_opt-step_response}
					\caption{Реакция на единичное воздействие}
				\end{figure}
				Время переходного процесса $t_{\text{пп}}=3T_{\mu}=0.015\,c$ и перерегулирование $\Delta h=0\%$
			\subsubsection{Диаграммы Боде}
				\begin{figure}[H]
					\centering
					\subfloat{\includegraphics[width=0.5\textwidth]{linear_opt-bode-raz}}
					\subfloat{\includegraphics[width=0.5\textwidth]{linear_opt-bode-zam}}
					\caption{Диаграммы Боде линейного оптимума}
				\end{figure}
			\subsubsection{Реакции на входное воздействие}
				\begin{figure}[H]
					\centering
					\subfloat{\includegraphics[width=0.5\textwidth]{linear_opt-const}}
					\subfloat{\includegraphics[width=0.5\textwidth]{linear_opt-lin}}\\
					\centering\subfloat{\includegraphics[width=0.5\textwidth]{linear_opt-quad}}
					\caption{Реакция на различные воздействия}
				\end{figure}
		\subsection{Технический оптимум}
			\subsubsection{Переходная функция}
				$$W(s)=\frac{1}{2T_\mu^2 s^2+2T_\mu s+1}$$
				\begin{figure}[H]
					\centering
					\includegraphics{technical-step_response}
					\caption{Реакция на единичное воздействие}
				\end{figure}
				Время переходного процесса $t_{\text{пп}}=4.1T_{\mu}=0.021\,c$ и перерегулирование $\Delta h=4.3\%$
			\subsubsection{Диаграммы Боде}
				\begin{figure}[H]
					\centering
					\subfloat{\includegraphics[width=0.5\textwidth]{technical-bode-raz}}
					\subfloat{\includegraphics[width=0.5\textwidth]{technical-bode-zam}}
					\caption{Диаграммы Боде технического оптимума}
				\end{figure}
			\subsubsection{Реакции на входное воздействие}
				\begin{figure}[H]
					\centering
					\subfloat{\includegraphics[width=0.5\textwidth]{technical-const}}
					\subfloat{\includegraphics[width=0.5\textwidth]{technical-lin}}\\
					\centering\subfloat{\includegraphics[width=0.5\textwidth]{technical-quad}}
					\caption{Реакция на различные воздействия}
				\end{figure}
		\subsection{Симметричный оптимум}
			\subsubsection{Переходная функция}
				$$W(s)=\frac{4T_\mu s+1}{8T_\mu^3 s^3+8T_\mu^2 s^2+4T_\mu s+1}$$
				\begin{figure}[H]
					\centering
					\includegraphics{symmetrical-step_response}
					\caption{Реакция на единичное воздействие}
				\end{figure}
				Время переходного процесса $t_{\text{пп}}=14.7T_{\mu}=0.073\,c$ и перерегулирование $\Delta h=43.39\%$
			\subsubsection{Диаграммы Боде}
				\begin{figure}[H]
					\centering
					\subfloat{\includegraphics[width=0.5\textwidth]{symmetrical-bode-raz}}
					\subfloat{\includegraphics[width=0.5\textwidth]{symmetrical-bode-zam}}
					\caption{Диаграммы Боде технического оптимума}
				\end{figure}
			\subsubsection{Реакции на входное воздействие}
				\begin{figure}[H]
					\centering
					\subfloat{\includegraphics[width=0.5\textwidth]{symmetrical-const}}
					\subfloat{\includegraphics[width=0.5\textwidth]{symmetrical-lin}}\\
					\centering\subfloat{\includegraphics[width=0.5\textwidth]{symmetrical-quad}}
					\caption{Реакция на различные воздействия}
				\end{figure}
		\subsection{Биномиальный оптимум}
			\subsubsection{Переходная функция}
				$$W(s)=\frac{1}{3T_\mu^2 s^2+3T_\mu s+1}$$
				\begin{figure}[H]
					\centering
					\includegraphics{binomial-step_response}
					\caption{Реакция на единичное воздействие}
				\end{figure}
				Время переходного процесса $t_{\text{пп}}=6.6T_{\mu}=0.033\,c$ и перерегулирование $\Delta h=0.43\%$
			\subsubsection{Диаграммы Боде}
				\begin{figure}[H]
					\centering
					\subfloat{\includegraphics[width=0.5\textwidth]{binomial-bode-raz}}
					\subfloat{\includegraphics[width=0.5\textwidth]{binomial-bode-zam}}
					\caption{Диаграммы Боде биномиального оптимума}
				\end{figure}
			\subsubsection{Реакции на входное воздействие}
				\begin{figure}[H]
					\centering
					\subfloat{\includegraphics[width=0.5\textwidth]{binomial-const}}
					\subfloat{\includegraphics[width=0.5\textwidth]{binomial-lin}}\\
					\centering\subfloat{\includegraphics[width=0.5\textwidth]{binomial-quad}}
					\caption{Реакция на различные воздействия}
				\end{figure}
		\subsection{Астатизм третьего порядка}
			\subsubsection{Переходная функция}
				$$W(s)=\frac{(16T_\mu s+1)(4T_\mu s+1)}{128T_\mu^4 s^4+128T_\mu^3s^3+64T_\mu^2 s^2+20T_\mu s+1}$$    
				\begin{figure}[H]
					\centering
					\includegraphics{astatism-step_response}
					\caption{Реакция на единичное воздействие}
				\end{figure} 
				Время переходного процесса $t_{\text{пп}}=14.7T_{\mu}=0.073\,c$ и перерегулирование $\Delta h=56.13\%$
			\subsubsection{Диаграммы Боде}
				\begin{figure}[H]
					\centering
					\subfloat{\includegraphics[width=0.5\textwidth]{astatism-bode-raz}}
					\subfloat{\includegraphics[width=0.5\textwidth]{astatism-bode-zam}}
					\caption{Диаграммы Боде астатизма третьего порядка}
				\end{figure}
			\subsubsection{Реакции на входное воздействие}
				\begin{figure}[H]
					\centering
					\subfloat{\includegraphics[width=0.5\textwidth]{astatism-const}}
					\subfloat{\includegraphics[width=0.5\textwidth]{astatism-lin}}\\
					\centering\subfloat{\includegraphics[width=0.5\textwidth]{astatism-quad}}
					\caption{Реакция на различные воздействия}
				\end{figure}
		\subsection{Сравнение характеристик оптимумов}
			В ходе работы получены следующие величины времени переходного процесса $t_{\text{пп}}$ и перерегулирования $\Delta h$.
			\begin{center}
				\begin{tabular}{|c|c|c|c|c|c|}
					\hline
					& Линейный & Технический & Симметричный & Биномиальный & Астатизм \\
					\hline
					$t_{\text{пп}}$ & $3T_{\mu}=0.015\,c$ & $4.1T_{\mu}=0.021\,c$ & $14.7T_{\mu}=0.073\,c$ & $6.6T_{\mu}=0.033\,c$ & $16.35T_{\mu}=0.082\,c$ \\
					\hline
					$\Delta h$ & 0\% & 4.3\% & 43.39\% & 0.43\% & 56.13\% \\
					\hline
				\end{tabular}	
			\end{center}
		\subsection{Порядок астатизма оптимумов}
            Для определения порядка астатизма приведем графики ошибок систем при постоянном воздействии.
            \begin{center}
                \begin{figure}[H]
                    \includegraphics[width=\linewidth, trim={1cm 1cm 1cm 0.5cm}, clip]{compare-errors}
                    \caption{Сравнение ошибок систем при разных воздействиях}
                \end{figure}
            \end{center}
            
            Если ошибка по окончанию переходных процессов для воздействия $n\mbox{-го}$ порядка равна нулю, то система имеет астатизм как минимумм $(n+1)\mbox{-го}$ порядка. Если ошибка имеет постоянное ненулевое значение, тогда система имеет астатизм $n\mbox{-го}$ порядка.
            
            Таким образом на основании полученных данных:
            \begin{center}
                \begin{tabular}{|c|c|c|c|c|c|}
                    \hline
                    & Линейный & Технический & Симметричный & Биномиальный & Астатизм \\
                    \hline
                    Порядок астатизма & 1 & 1 & 2 & 1 & $\ge 3$ \\
                    \hline
                \end{tabular}	
            \end{center}
        
    \section{Вывод}
        В ходе моделирования систем были эксперементально получены значения перерегулирования и времени переходных процессов для данных оптимумов. Получены ЛАЧХ и ФЧХ для систем различной настройки, а также определён порядок астатизма, получающийся в результате настройки системы.
        
        Очевидно, что при увеличении порядка астатизма реакция системы становится более <<колебательной>> --- растёт перерегулирование и время переходного процесса. При этом системы с б\'{о}льшим порядком астатизма отрабатывают воздействия с установившейся ошибкой, чем системы с меньшим порядком.
\end{document}
