\documentclass[a4paper, 12pt]{article}

\usepackage{cmap}
\usepackage[T2A]{fontenc}
\usepackage[utf8]{inputenc}
\usepackage[english, russian]{babel}

% Для математики
\usepackage{amsmath}

% Для картинок
\usepackage{graphicx}
\usepackage{float}
\usepackage{wrapfig}
\usepackage[caption=false]{subfig}

% Оформление документа
\usepackage{indentfirst}
\usepackage{geometry}
\geometry{right=20mm}
\geometry{left=20mm}
\geometry{top=20mm}
\geometry{bottom=20mm}

% Пути до файлов
\graphicspath{{../src/img/}}

% Вставки кода
\usepackage{pythontex}
\pyc{data_file = '../src/data.txt'} % Путь до файла с дано
\begin{pycode}
with open(data_file, 'r') as data:
    K1, K2, T1 = data.readline().split()
print(f'$K_1={K1}$', f'$K_2={K2}$', f'$T_1={T1}$', sep=r'\\')
\end{pycode}


\begin{document}
    \begin{titlepage}
	\begin{center}
		\textit{МИНИСТЕРСТВО ОБРАЗОВАНИЯ И НАУКИ\\
			РОССИЙСКОЙ ФЕДЕРАЦИИ}
		\vspace{1ex}

		федеральное государственное бюджетное образовательное учреждение\\
		высшего профессионального образования
		\vspace{1ex}

		\textbf{САНКТ-ПЕТЕРБУРГСКИЙ НАЦИОНАЛЬНЫЙ ИССЛЕДОВАТЕЛЬСКИЙ УНИВЕРСИТЕТ ИТМО}
		\vspace{13ex}

		Лабораторная работа №4\\
		<<Анализ и моделирование систем с цифровым ПД-регулятором>>\\
		по дисциплине <<Компьютерное управление мехатронными системами>>\\
	\end{center}
	\vspace{15em}
	\begin{flushright}
		\noindent
		Выполнил:\\
		студент гр. R3425\\
		Борисов М. В.\\
		Преподаватель:\\
		Ловлин С. Ю.
	\end{flushright}
	\vfill
	\begin{center}
		\largeСанкт-Петербург\\
		2020 г.\\
	\end{center}
\end{titlepage}



    \section{Цель работы}
    \begin{enumerate}
        \item Исследовать принцип работы цифрового П-регулятора
        \item Исследовать переходный процесс при периоде дискретезации $T_0\leq0.1T_\mu$
        \item Исследовать переходный процесс при периоде дискретезации $T_0=T_1$
        \item Исследовать влияние запаздывания на переходный процесс
        \item Компенсировать запаздывание системы
    \end{enumerate}


    \section{Дано}
    \begin{figure}[H]
        \centering\includegraphics{model-digital_example}
        \caption{Система исследования цифрового П-регулятора}
    \end{figure}
    \begin{figure}[H]
        \centering\includegraphics[width=\linewidth, height=\textheight, keepaspectratio]{model-p_isled}
        \caption{Система исследования переходных процессов}
    \end{figure}
    \begin{figure}[H]
        \centering\includegraphics[width=\linewidth, height=\textheight, keepaspectratio]{model-p_isled_zap}
        \caption{Система исследования запаздывания}
    \end{figure}

    \begin{center}
        \noindent\pys{$K_1=!{K1},\,K_2=!{K2},\,T_1=!{T1}$}\\
    \end{center}


    \section{Выполнение работы}

    \subsection{Исследование цифрового П-регулятора}
    \begin{figure}[H]
        \centering\includegraphics{p_regul-1}
        \caption{Сравнение аналогового и цифрового П-регулятора}
    \end{figure}

    \subsection{Вывод П-регулятора}
    Допустим производится настройка системы на технический оптимум.\\
    $W_{\mbox{оу}}(s)=\dfrac{K_1K_2}{s(T_1s+1)}$ --- передаточная функция объекта управления\\
    $W_{\mbox{рс}}(s)=\dfrac{1}{2T_\mu s(T_\mu s+1)}$ --- передаточная функция разомкнутой системы.\\

    \noindentПримем $T_1=T_\mu$, тогда передаточная функция регулятора
    \[
        W_{\mbox{рег}}(s)=\dfrac{W_{\mbox{рс}}(s)}{W_{\mbox{оу}}(s)}=
        \dfrac{\cfrac{1}{2T_\mu s(T_\mu s+1)}}{\cfrac{K_1 K_2}{s(T_\mu s+1)}}=
        \dfrac{1}{2T_\mu K_1 K_2}
    \]

    \subsection{Случай $T_0 \leq 0.1T_\mu$}
    \begin{figure}[H]
        \centering\includegraphics{p_regul-2}
        \caption{Сравнение аналогового и цифрового П-регулятора}
    \end{figure}

    \subsection{Случай $T_0 = T_1$}
    \begin{figure}[H]
        \centering\includegraphics{p_regul-3}
        \caption{Сравнение аналогового и цифрового П-регулятора}
    \end{figure}

    \subsection{Запаздывание аналогового сигнала}
    \begin{figure}[H]
        \centering\includegraphics{p_regul-4}
        \caption{Сравнение аналогового и цифрового П-регулятора}
    \end{figure}

    \subsection{Компенсация запаздывания}
    \begin{figure}[H]
        \centering\includegraphics{p_regul-5}
        \caption{Сравнение аналогового и цифрового П-регулятора}
    \end{figure}
\end{document}
