\documentclass[a4paper, 12pt]{article}

\usepackage{cmap}
\usepackage[T2A]{fontenc}
\usepackage[utf8]{inputenc}
\usepackage[english, russian]{babel}

% Для математики
\usepackage{amsmath}

% Для картинок
\usepackage{graphicx}
\usepackage{float}
\usepackage{wrapfig}
\usepackage[caption=false]{subfig}

% Оформление документа
\usepackage{indentfirst}
\usepackage{geometry}
\geometry{right=20mm}
\geometry{left=20mm}
\geometry{top=20mm}
\geometry{bottom=20mm}

% Пути до файлов
\graphicspath{{../src/img/}}

% Вставки кода
\usepackage[gobble=auto]{pythontex}
\pyc{data_file = '../src/data.txt'} % Путь до файла с дано
\begin{pycode}
with open(data_file, 'r') as data:
    K1, K2, T1 = data.readline().split()
\end{pycode}

\begin{document}
    \begin{titlepage}
	\begin{center}
		\textit{МИНИСТЕРСТВО ОБРАЗОВАНИЯ И НАУКИ\\
			РОССИЙСКОЙ ФЕДЕРАЦИИ}
		\vspace{1ex}
		
		федеральное государственное бюджетное образовательное учреждение\\
		высшего профессионального образования
		\vspace{1ex}
		
		\textbf{САНКТ-ПЕТЕРБУРГСКИЙ НАЦИОНАЛЬНЫЙ ИССЛЕДОВАТЕЛЬСКИЙ УНИВЕРСИТЕТ ИТМО}
		\vspace{13ex}
		
		Лабораторная работа №1\\
		<<Оптимумы линейных систем>>\\
		по дисциплине <<Компьютерное управление мехатронными системами>>\\
	\end{center}
	\vspace{15em}
	\begin{flushright}
		\noindent
		Выполнил:\\
		студент гр. R3425\\
		Борисов М. В.\\
		Преподаватель:\\
		Ловлин С. Ю.
	\end{flushright}
	\vfill
	\begin{center}
		\largeСанкт-Петербург\\
		2020 г.\\
	\end{center}
\end{titlepage}
    \setcounter{page}{2}

    \section{Цель работы}
        \begin{enumerate}
            \item Исследовать принцип работы цифрового ПД--регулятора
            \item Исследовать компенсацию постоянной времени апериодического звена с помощью ПД--регулятора
            \item Синтезировать систему с ОУ и ПД--регулятором
            \item Показать эквивалентность аналоговой и цифровой систем
        \end{enumerate}

    \section{Дано}
        \begin{center}
            \noindent\pys{$K_1=!{K1},\,K_2=!{K2},\,T_1=!{T1}$}\\
        \end{center}

    \section{Выполнение работы}
        \subsection{Исследование цифрового ПД--регулятора}
            Цифровой ПД--регулятор отличается от аналогового наличием времени дискретизации $T_0$,
            определяющим характеристики регулятора.
            \begin{figure}[H]
                \centering\includegraphics[width=\linewidth, height=\textheight, keepaspectratio]{model-d_analog}
                \caption{Система исследования цифрового ПД--регулятора}
            \end{figure}
            \begin{figure}[H]
                \centering\includegraphics[scale=1]{pd_regul-1}
                \caption{Сравнение аналогового и цифрового ПД--регулятора}
                
            \end{figure}
            На графике видно, что идеальный аналоговый ПД--регулятор имеет реакцию в виде мгновенного импульса ($ \delta $ -- функции) в момент скачка задания после чего сразу принимает установившееся значение согласно пропорциональному коэффициенту.
            Цифровой ПД--регулятор, в свою очередь, в момент скачка задания не имеет мгновенного импульса, но принимает некоторое конечное значение на период дискретизации. По прошествию этого времени ПД--регулятор принимает установившееся значение.
        
        \subsection{Компенсация постоянной времени}
            \begin{figure}[H]
                \centering\includegraphics[width=\linewidth, height=\textheight, keepaspectratio]{model-d_analog_komp}
                \caption{Система исследования компенсации постоянной времени}
            \end{figure}
            Проведём моделирование системы с последовательным соединением ПД--регулятора и апериодического звена с постоянной времени $ T_1 $. Примем $ K_{pa}=K_p=1 $ и $ K_{da} = K_d = T_1 $, время дискретизации $ T_0 = 0.1$.
            \begin{figure}[H]
                \centering\includegraphics[scale=0.8]{pd_regul-2}
                \caption{Сравнение аналогового и цифрового ПД--регулятора}
            \end{figure}
            На графике видно, что при данных условиях реакции систем практически совпадают и требуют некоторого времени до достижения установившегося значения. Что означает наличие некоторой нескомпенсированной постоянной времени.

        \subsection{Методы расчёта $ K_d $}
            \begin{figure}[H]
                \centering\includegraphics[width=\linewidth, height=\textheight, keepaspectratio]{model-d_analog_eq}
                \caption{Система исследования методов расчёта $ K_d $}
            \end{figure}
            Для компенсации остаточной постоянной времени необходимо уточнить коэффициенты регулятора.
            Коэффициент дифференцирующей составляющей цифрового ПД--регулятора $ K_d $ можно посчитать зная коэффициент $ K_{da} $ аналогового ПД--регулятора и время дискретизации $ T_0 $ двумя методами:
            
            $$  K_d = \dfrac{K_{da}}{T_0}  $$ 
            $$  K_d = \dfrac{1}{\exp\left(\dfrac{T_0}{K_{da}}\right)-1}  $$
            \begin{figure}[H]
                \centering\includegraphics[scale=0.8]{pd_regul-3}
                \caption{Сравнение аналогового и цифрового ПД--регулятора при $ K_d = \dfrac{K_{da}}{T_0} $}
            \end{figure}
            \begin{figure}[H]
                \centering\includegraphics[scale=0.8]{pd_regul-4}
                \caption{Сравнение аналогового и цифрового ПД--регулятора при $ K_d = \dfrac{1}{\exp\left(\frac{T_0}{K_{da}}\right)-1} $}
            \end{figure}
            По графикам видно, что оба метода компенсируют остаточную постоянную времени (не считая время дискретизации). Однако первый способ менее точен, из-за чего возникает значительное перерегулирование сигнала и требуется время до достижения установившегося значения.
            
            Второй способ напротив не имеет перерегулирования и реагирует сразу (за один период дискретизации), опережая аналоговый регулятор.
            
        \subsection{Синтез системы с ПД регулятором}
            \subsubsection{Вывод ПД--регулятора}
                Допустим производится настройка системы на технический оптимум.\\
                $W_{\mbox{оу}}(s)=\dfrac{K_1 K_2}{s(T_1s+1)}$ --- передаточная функция объекта управления\\
                $W_{\mbox{рс}}(s)=\dfrac{1}{2T_\mu s(T_\mu s+1)}$ --- передаточная функция разомкнутой системы.\\
            
                \[
                    W_{\mbox{рег}}(s)=\dfrac{W_{\mbox{рс}}(s)}{W_{\mbox{оу}}(s)}=
                    \dfrac{\cfrac{1}{2T_\mu s(T_\mu s+1)}}{\cfrac{K_1 K_2}{s(T_1s+1)}}=
                    \dfrac{1}{2T_\mu K_1 K_2}\dfrac{T_1 s + 1}{T_\mu s +1}
                \]
                
            \subsubsection{Синтез системы без запаздывания аналогового сигнала}
                Составим модель в которой будем управлять объектом с помощью ПД--регулятора.
                \begin{figure}[H]
                    \centering\includegraphics[width=\linewidth, height=\textheight, keepaspectratio]{model-d_analog_ss}
                    \caption{Система без запаздывания аналогового сигнала}
                \end{figure}
                \begin{figure}[H]
                    \centering\includegraphics[scale=0.8]{pd_regul-5}
                    \caption{Сравнение аналогового и цифрового ПД--регулятора}
                \end{figure}
                На графике видно существенную разницу в реакциях аналоговой и цифровой систем. Цифровой регулятор имеет более колебательный, длительный переходный процесс и заметно большее перерегулирование чем аналоговый, что связано с запаздыванием из-за наличия времени дискретизации.

            \subsubsection{Синтез системы с запаздыванием аналогового сигнала}
                Для компенсации запаздывания проделаем те же операции, что в предыдущих работах -- внесём запаздывание в аналоговую систему и компенсируем его с учётом $ T_\mu = T_{\mu r} + \dfrac{T_0}{2} $.
                \begin{figure}[H]
                    \centering\includegraphics[width=\linewidth, height=\textheight, keepaspectratio]{model-d_analog_ss_zap}
                    \caption{Система с запаздыванием аналогового сигнала}
                \end{figure}
                \begin{figure}[H]
                    \centering\includegraphics[scale=0.8]{pd_regul-6}
                    \caption{Сравнение аналогового и цифрового ПД--регулятора}
                \end{figure}
                На графике видно, что после компенсации запаздывания реакции систем снова стали совпадать.

    \section{Вывод}
        В работе был изучен цифровой ПД--регулятор, его характеристики и отличия от аналогового регулятора.
        
        Изучены различные методы расчёта коэффициента дифференцирующей составляющей цифрового ПД--регулятора и показано, что метод с экспонентой является более точным, однако его использование может быть ограничено точностью вычислений цифровой системы. В таком случае возможно применение более простого соотношения $ K_d = \dfrac{K_{da}}{T_0} $.
        
        Изучено явление запаздывания сигнала и возможность компенсации с помощью представления звена запаздывания как
        апериодического первого порядка.
        
        Также было показано, что возможно синтезировать систему с цифровым ПД--регулятором эквивалентную
        аналоговой с заданными параметрами, однако прямо использовать коэффициенты аналоговой системы в цифровой в данном случае нельзя и необходимо пользоваться одним из рассмотренных соотношений для расчёта коэффициентов.
\end{document}
